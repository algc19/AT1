% Lecture 16

\lecture[The core of the representability argument.]{2021-12-07}

Given the above discussion, for all CW-complexes $X$, the set $[X,K(A,n)]$ becomes an abelian group by $[f]+[g]=[\mu\circ(f,g)]$.

\begin{remark}\todo{I don't really understand this remark}
One can realize a $K(A,n)$ as a topological abelian group, e.g. $|\til A[\Delta^n/\de]|$. $\til A[\Delta^n/\de]$ is a simplicial abelian group, and $|-|:\sSet\to\Top_\text{cpt. gen.}$ commutes with products, which implies that $|\til A[\Delta^n/\de]|$ is an abelian topological group with product given by:
\[|\til A[\Delta^n/\de]|^2\xto{\cong}|\til A[\Delta^n/\de]\times\til A[\Delta^n/\de]|\xto{|\mu|}|\til A[\Delta^n/\de]|.\]
\end{remark}

\begin{example}\todo[color=red]{I really don't understand this example}
We know that $\rp{\infty}=K(\Z/2,1)$ and $\cp{\infty}=K(\Z,2)$. We also talked about how $\rp{\infty}$ classifies principal $\Z/2$-bundles (i.e. two-fold coverings). Then for every paracompact space $X$, we have:
\[
\begin{tikzcd}
    \left[X,\rp{\infty}\right] \ar[r,"\cong"] & \operatorname{Prin}_{\Z/2}(X) \ar[r,"\cong"] & \Pic_\R(X)\\[-4ex]
    f \ar[rr,mapsto] & & f^* (j)\\[-4ex]
    & E\to X \ar[r,mapsto] & E\times_{C_2}\R\to X\\[-4ex]
    & (F-s_0(X))/\R_{>0}\to X & F\to X \ar[l,mapsto]
\end{tikzcd}
\]
where $j$ is the tautological line bundle on $\rp{\infty}$ and $s_0$ is the zero section. Under this bijection, the group operation on the left corresponds to the tensor product on the right (?).

Similarly we have:
\[[X,\cp{\infty}]\xto{\cong}\Pic_\CC(X)\]
sending $[f:X\to\cp{\infty}]$ to pullback of the universal complex line bundle over $\cp{\infty}$ and again with addition corresponding to the tensor product of line bundles (?).
\end{example}

\subsection{The Fundamental Cohomology Class}

\begin{propdef}[Fundamental cohomology class]\label{propdef:fundamental-cohomology-class}
Let $(X,\phi)$ be an EM-space of type $(A,n)$, $n\ge1$, $A$ abelian. Then there is a unique class $\iota\in H^n(X;A)$ such that the composite:
\[\pi_n(X,x)\xto{\cong}H_n(X;\Z)\xto{\Phi(\iota)}A\]
is the isomorphism $\phi:\pi_n(X,x)\to A$, where $\Phi:H^n(X;A)\to\Hom(H_n(X;\Z),A)$ is the homomorphism from the universal coefficient theorem.
\end{propdef}

\begin{proof}
Since $X$ is $(\ni)$-connected, the Hurewicz homomorphism $h:\pi_n(X,x)\to H_n(X,\Z)$ is an isomorphism; for $n=1$ this is true because $A$ is abelian. Because $H_\ni(X,\Z)$ is trivial (for $n\ge2$) or free (if $n=1$), the $\Ext$-term in the UCT vanishes, so:
\[\Phi:H^n(X;A)\xto{\cong}\Hom(H_n(X;\Z),A)\]
is an isomorphism. Then we can define $\iota=\Phi^\inv(\phi\circ h^\inv)$.
\end{proof}

We want to see that for a CW-complex $Y$ the fundamental class $\iota\in H^n(K(A,n);A)$ gives rise to a natural isomorphism:
\[
[Y,K(A,n)]\to H^n(Y;A),\ [f:Y\to K(A,n)]\mapsto f^*(\iota).
\]

The convention for $n=0$ is that $K(A,0)$ is just the group $A$ with the discrete topology, the map $\mu:K(A,0)^2\to K(A,0)$ is the addition on $A$ and $i: K(A,0)\to K(A,0)$ is the inverse map, while $\iota\in H^0(A,A)$ is represented by the identity cocycle.

\begin{theorem}
For all $n\ge0$ and all abelian groups $A$, the evaluation map at $\iota$ is a group morphism.
\end{theorem}

\begin{proof}
Given the convention above, without loss of generality we can assume $n\ge1$.

We start with a \enquote{universal example}, from which the general case will easily follow. Let $Y=K(A,n)\times K(A,n)$ and $p_1,p_2:K(A,n)^2\to K(A,n)$ the projections. Observe that the sum $[p_1]+[p_2]$ is represented by the map $\mu$ realizing group addition:
\[
K(A,n)\times K(A,n)\xto{(p_1,p_2)=\id}K(A,n)\times K(A,n)\xto{\mu}K(A,n),\]
\[\text{i.e. } [p_1]+[p_2]=[\mu\circ(p_1,p_2)]=[\mu].
\]\ 

Then we want to show that in the special case of $[\mu]=[p_1]+[p_2]$ evaluation at $\iota$ is additive:
\[\mu^*(\iota)=p_1^*(\iota)+p_2^*(\iota).\]
We will use the following isomorphism:
{\small
\[ H^n(K(A,n)^2;A)\xto{\cong}\Hom(H_n(K(A,n)^2;\Z),A)\xto{\cong}\Hom(\pi_n(K(A,n)^2,*),A)\xto{\cong}\Hom(A\times A,A)\]}
where the first isomorphism comes from the UCT and the second is the Hurewicz isomorphism. Under this isomorphism $\mu^*(\iota)$ corresponds to the addition $A\times A\to A$ and the projections to the corresponding projections $A\times A\to A$. But in $\Hom(A\times A,A)$ clearly we do have that the sum of the projections is the addition, hence the same holds in $H^n(K(A,n)^2;A)$.

Now for the general case, let $Y$ be an arbitrary space and $f,g:Y\to K(A,n)$ continuous maps. Then we have:
\begin{align*}
(f+g)^*(\iota)=(\mu\circ(f,g))^*(\iota)=(f,g)^*(\mu^*(\iota))&=(f,g)^*(p_1^*(\iota)+p_2^*(\iota))\\
&=(f,g)^*(p_1^*(\iota))+(f,g)^*(p_2^*(\iota))\\
&=f^*(\iota)+g^*(\iota).
\end{align*}
\end{proof}

\begin{lemma}
For all $n\ge1$, all abelian groups $A$ and all based CW-complexes $Y$, the forgetful map
\[[Y,K(A,n)]_*\to [Y,K(A,n)]\]
is a bijection.
\end{lemma}

\begin{proof}
We start by showing surjectivity. We will use that for $n\ge1$ any $K(A,n)$ is path-connected and that the inclusion of the basepoint $\cb{y}\into Y$ has the HEP\rightnote{Note (for those who studied the HEP long ago): clearly not every inclusion of a point has the HEP, this works (as always in this course) because $Y$ is a CW-complex.}. In particular, given any continuous map $f:Y\to K(A,n)$, we can choose a path $w:[0,1]\to K(A,n)$ from $f(y)$ to $x\in K(A,n)$. The HEP for $(\cb{y},Y)$ lets us choose a homotopy $H:Y\times[0,1]\to K(A,n)$ starting with $f$ and such that $H(y,-)=w$. Then $g=H(-,1)$ is freely homotopic to $f$ and based, hence the forgetful map sends $[g]$ to $[f]$.

To show injectivity, we first consider the case $n=1$. Let $f,g:Y\to K(A,1)$ be based maps that are freely homotopic. Then we have:
\[f_*=g_*:H_1(Y;\Z)\to H_1(K(A,1);\Z).\]
Since $A$ is abelian, the map $\pi_1(K(A,1),x)\to H_1(K(A,1),\Z)$ is an isomorphism, hence we have:
\[f_*=g_*:\pi_1(Y,y)\to \pi_1(K(A,1),x).\]
By theorem \ref{theorem:correspondence-maps-and-group-morphisms}\leftnote{In principle, theorem \ref{theorem:correspondence-maps-and-group-morphisms} would require $Y$ to be connected, but (Xiaoxiang Zhou suggested) it is easy to show that when $Y$ is not connected the result still holds: the idea is that a pointed map from a disjoint union\\ of connected components can\\ be decomposed as maps from each component, only one of which has to preserve the basepoint.}, $f$ and $g$ are then based homotopic.

Now let $n\ge2$. We will exploit the fact that a $K(A,n)$ is then simply connected. Let $Z$ be (more generally) any simply-connected space, $f,g:Y\to Z$ two based continuous maps that are freely homotopic. Let $H:Y\times[0,1]\to Z$ be a free homotopy from $f$ to $g$. Then $w=H(y,-):[0,1]\to Z$ is a loop at the basepoint of $Z$. Since $Z$ is simply-connected, this loop is homotopic to the constant loop at the endpoint, relative to it, say by a homotopy $G:[0,1]\times[0,1]\to Z$. The HEP of the pair $(Y\times[0,1],Y\times\cb{0}\cup\cb{y}\times[0,1]\cup Y\times\cb{1})$ applied to
\[K:(Y\times\cb{0}\cup\cb{y}\times[0,1]\cup Y\times\cb{1})\times[0,1]\xto{\const_f\cup\,G\,\cup\, \const g} Z\]
yields a map $\bar K:Y\times[0,1]\times[0,1]\to Z$ extending $K$ and such that $\bar K(-,-,0)=H$. Then the map $H'=\bar K(-,-,1):Y\times[0,1]\to Z$ is a based homotopy between $f$ and $g$.
\end{proof}

We are now ready to prove the result we promised.

\begin{theorem}
For all $n\ge0$, all abelian groups $A$ and all CW-complexes $Y$, the evaluation map
\[
    [Y,K(A,n)]\to H^n(Y;A),\ [f]\mapsto f^*(\iota)
\]
is a group isomorphism.
\end{theorem}

\begin{proof}\renewcommand{\qedsymbol}{\textit{To be continued...}}
If $n=0$, we defined $K(A,0)$ as $A$ with the discrete topology, so clearly
\[[Y,K(A,n)]=\Hom_\Set(\pi_0(Y),A)\cong H^0(Y;A)\]
with addition in $[Y,K(A,n)]$ corresponding to pointwise addition in $\Hom_\Set(\pi_0(Y),A)$.

Now let $n\ge1$. The theorem holds for $Y=\emptyset$ (since both sides are then $0$), so without loss of generality we can assume $Y$ is non-empty. We choose a basepoint $y\in Y$. We will start by proving the theorem in a special case, which will then be used to conclude in generality.

Special case. Suppose $Y$ is $(n-1)$-connected. Then $H_\ni(Y;\Z)$ is trivial (if $n\ge2$, by Hurewicz) or at least free (if $n=1$), hence $\Ext(H_\ni(Y;\Z),A)$ vanishes and the morphism $\Phi:H^n(Y;A)\to\Hom(H_n(Y;\Z),A)$ from the UCT is an isomorphism. We also have that for $n\ge2$ the Hurewicz map $h:\pi_n(Y,y)\to H_n(Y;\Z)$ is an isomorphism, while for $n=1$ it is the universal homomorphism into the abelianization. In both cases precomposition yields an isomorphism
\[\Hom(h,A):\Hom_\Grp(H_n(Y;\Z),A)\xto{\cong}\Hom_\Grp(\pi_n(Y,y),A).\]
The composite
\[[Y,K(A,n)]_*\xto{[f]\mapsto f^*(\iota)} H^n(Y;A)\underset{\cong}{\xto{\Phi}}\Hom(H_n(Y;\Z),A)\underset{\cong}{\xto{\Hom(h,A)}}\Hom_\Grp(\pi_n(Y,y),A)\]
sends $[f]$ to $\pi_n(f):\pi_n(Y,y)\to\pi_n(K(A,n),*)\to A$, hence we know that it is a bijection by theorem \ref{theorem:correspondence-maps-and-group-morphisms} (since $Y$ is a $(n-1)$-connected CW-complex and $K(A,n)$ has vanishing homotopy groups for $k>n$). Since the composite, the second and the third maps are bijections, the valuation at $\iota$ map $[f]\mapsto f^*(\iota)$ must also be.
\end{proof}
