% Lecture 10

\lecture[Legend has it that if you repeat compact-open for two hours loop spaces will appear.]{2021-11-17}

I missed this lecture, thanks to Paul for providing photos of the blackboard!

Up to now we have proved the following facts:
\begin{itemize}[label={-}]
    \item for a Hurewicz fibration (resp. Serre fibration) over a path-connected space, the fibres are homotopy equivalent (resp. the same, but whenever they admit CW-structures),
    \item fibre bundles, Hurewicz fibrations and Serre fibrations are stable under base change,
    \item if $X$ is compact and $Z$ is a metric space, the compact-open topology on $Z^X$ agrees with the topology of the supremum metric.
\end{itemize}

\begin{example}
Let $I$ be any set, endowed with the discrete topology. Let $Z$ be a space and consider $Z^I=\prod_I Z$. Then the compact-open topology on $Z^I$ agrees with the product topology.
\begin{itemize}
    \item Let $J$ be any subset of I; then $J$ is compact if and only if it is finite. So the sets $W(J,O)=\prod_{j\in J}O\times\prod_{j\not\in J}Z$ for $J$ finite and $O$ open form a subbasis of the compact-open topology. These sets are open in the product topology.
    \item A subbasis in the product topology is given by the sets $\prod_{j\in J}O_j\times\prod_{j\not\in J}Z$ for $J\subset I$ finite and $O_j$ open in $Z$. But then we have that
    \[\prod_{j\in J}O_j\times\prod_{j\not\in J}Z=\bigcap_{j\in J}(O_j\times\prod_{i\neq j}Z)=\bigcap_{j\in J}W(\cb{j},O_j)\]
    is open in the compact-open topology.
\end{itemize}
\end{example}

\begin{theorem}\label{theorem:compact-subbasis-topology}
Let $X$ and $Z$ be spaces and let $\S$ be a subbasis of the topology on $Z$. Then the sets $W(K,O)$ for $K\subset X$ compact and $O\in\S$ form a subbasis of the compact-open topology.
\end{theorem}

\begin{proof}
The "new topology" (generated by $W(K,O)$ for $O\in\S$) is clearly contained in the compact-open topology. We need to show that for all $O\subset Z$ open, the set $W(K,O)$ is in the new topology.

If $O\in\S$, this is true by definition.

If $O=\nns{O}{\cap}{n}$ with $O_i\in\S$, then
\[W(K,O)=W(K,O_1)\cap\cdots\cap W(K,O_n)\]
which is open in the new topology.

Now, suppose $O=\bigcup_{i\in I}O_i$ with $O_i\in\Bb$, where $\Bb$ is the basis generated by $\S$, i.e. the set of all finite intersections of sets in $\S$. Let $f\in W(K,O)$. Then for each $x\in X$ there is a set $O_x\in\Bb$ with $f(x)\in O_x\subset O$. Since $K$ is compact, hence locally compact, there is a compact neighbourhood $K_x$ of $x$ with $f(K_x)\subset O_x$. The covering $\cb{K_x}_{x\in K}$ of the compact set $K$ has a finite subcover $K\subset K_{x_1}\cup\cdots\cup K_{x_n}$. Then
\[f\in \bigcap_{i=1,\dots,n}W(K_{x_i},O_{x_i})\subset W(K,O).\]
\end{proof}

\begin{theorem}
Let $X$ and $Z$ be spaces and let $X$ be locally compact. Then the evaluation map $\ev:Z^X\times X\to Z,\ \ev(f,x)=f(x)$, is continuous.
\end{theorem}

\begin{proof}
Let $O$ be any open subset of $Z$. We want to show that
\[\ev^{-1}(O)=\cb{(f,x)\in Z^X\times X:f(x)\in O}\]
is open. Let $(f,x)\in Z^X\times X$ be in $\ev^{-1}(O)$, so $f(x)\in O$. Since $f$ is continuous, $f^{-1}(O)$ is an open neighbourhood of $x$. Since $X$ is locally compact, there is a compact neighbourhood $K$ of $x$ inside $f^{-1}(O)$. Then $f\in W(K,O)$ and $W(K,O)\times K$ is a neighbourhood of $(f,x)$ in $Z^X\times X$. Then $\ev(W(K,O)\times K)\subset O$, hence $(f,x)\in W(K,O)\times K\subset\ev^{-1}(O)$, so $\ev^{-1}$ is open.
\end{proof}

\begin{theorem}
Let $X,Y$ and $Z$ be spaces.
\begin{numerate}
    \setcounter{enumi}{-1}
    \item For every continuous map $f:X\times Y\to Z$, the adjoint map $\Phi(f):Y\to Z^X$, $\Phi(f)(y)(x)=f(x,y)$ is continuous. So $\Phi$ defines a map $Z^{X\times Y}\to(Z^X)^Y$.
    \item The map $\Phi:Z^{X\times Y}\to(Z^X)^Y$ is continuous.
    \item If $X$ is locally compact, then $\Phi$ is bijective.
    \item If $X$ is locally compact and $X$ and $Y$ are Hausdorff, then $\Phi$ is a homeomorphism.
\end{numerate}
\end{theorem}

\begin{proof}\ 

(0) Let $f:X\times Y\to Z$ be continuous and $\Phi(f):Y\to Z^X$ the adjoint map. To show that $\Phi(f)$ is continuous, it suffices to show that the sets $(\Phi(f))^{-1}(W(K,O))$ are open in $Y$ for all $K\in X$ compact, $O\in Z$ open. Let $y\in (\Phi(f))^{-1}(W(K,O))$, i.e. $f(K\times\cb{y})\subset O$, or $K\times\cb{y}\subset f^{-1}(O)$ which is open. By the tube lemma, there is a open neighbourhood $U$ of $y$ in $Y$ such that $K\times U\subset f^{-1}(O)$ or equivalently $\Phi(f)(U)\subset W(K,O)$. Hence $y\in U\subset(\Phi(f))^{-1}(W(K,O))$, so the latter set is open in $Y$.

(1) The compact-open topology on $Z^X$ has a subbasis consisting of the sets $W(K,O)$ for $K\subset X$ compact and $O\subset Z$ open. By theorem \ref{theorem:compact-subbasis-topology} the compact-open topology on $(Z^X)^Y$ has a subbasis of the form $W(K',W(K,O))$ for $K\subset X$ compact, $K'\subset Y$ compact, $O\subset Z$ open. But $\Phi^{-1}(W(K',W(K,O)))=W(K\times K',O)$ which is open in the compact-open topology on $Z^{X\times Y}$ because $K\times K'$ is again compact.

(2) On the set-theoretic level, every map $Y\to Z^X$ is of the form $\Phi(f)$ for some unique map $f:X\times Y\to Z$. So $\Phi$ is continuous by (1) and injective. We have to show that when $X$ is locally compact and $g:Y\to Z^X$ is continuous, $\Phi^{-1}(g)$ is continuous. But since the evaluation map is continuous by (0), the composite
\[\Psi(g):X\times Y\cong Y\times X\xto{g\times\id}Z^X\times X\xto{\ev}Z\]
is continuous, hence we have:
\[\Phi(\Psi(g))(y)(x)=\Psi(g)(x,y)=\ev(g(y),x)=g(y)(x)\text{ for all }x\in X,y\in Y\]
so that $\Phi(\Psi(g))=g$.

(3) Now we suppose that $X$ is locally compact Hausdorff and $Y$ is Hausdorff. As we saw in (1), the sets $W(K',W(K,O))$ with $K\subset X$ and $K'\subset Y$ compact, $O\subset Z$ open, form a subbasis on $(Z^X)^Y$. Then
\[\Phi^{-1}(W(K',W(K,O)))=W(K\times K',O)\]
so that we just need to show that the sets $W(K\times K',O)$ generate the compact-open topology on $Z^{X\times Y}$. This is the content of the following lemma.
\end{proof}

\begin{lemma}
Let $X$ and $Y$ be Hausdorff spaces and $Z$ any space. Then the compact-open topology on $Z^{X\times Y}$ is generated by the sets $W(K\times K',O)$ for all $K\subset X$ compact, $K'\subset Y$ compact and $O\subset Z$ open.
\end{lemma}

\begin{proof}
Let $L$ be any compact subset of $X\times Y$ and $O$ an open subset of $Z$. We need to show that $W(L,O)$ is open in the potentially small topology generated by the sets $W(K\times K',O)$. Let $f\in W(L,O)$ be arbitrary, we have $L\subset f^{-1}(O)$, which is open in $X\times Y$. For each $(x,y)\in L$ we can choose $U_{x,y}$ open in $X$ and $V_{x,y}$ open in $V$ with $(x,y)\in U_{x,y}\times V_{x,y}\subset f^{-1}(O)$. We set $L_X=\pr_X(L)$ and $L_Y=\pr_Y(L)$, which are quasi-compact subsets of $X$ and $Y$ respectively. Since $X$ and $Y$ are Hausdorff, $L_X$ and $L_Y$ are indeed compact, hence locally compact. For each $(x,y)\in L$ we can find compact neighbourhoods $K_{x,y}$ of $x$ in $L_X$ and $K'_{x,y}$ of $y$ in $L_Y$. Then the sets $\cb{K_{x,y}\times K'_{x,y}}_{(x,y)\in L}$ cover L. Since $L$ is compact, there is a finite subcover $\cb{K_{x,y}\times K'_{x,y}}_{(x,y)\in I}$ with $I$ finite. Then
\[f\in\underbrace{\bigcap_{(x,y)\in I}W(K_{x,y}\times K'_{x,y},O)}_{\text{open in the potentially smaller topology}}=W(\bigcup_{(x,y)\in I}K_{x,y}\times K'_{x,y},O)\subset W(L,O)\]
where the last inclusion follows from $L\subset \bigcup_{(x,y)\in I}K_{x,y}\times K'_{x,y}$.
\end{proof}

\begin{remark}
The assignment $(Z,X)\mapsto Z^X$ is a contravariant functor for continuous maps in Hausdorff spaces and a covariant functor for continuous map in all spaces, i.e.
\[(-)^{(-)}:\Top\times\Top^\op_{H}\to \Top\]

Contravariant functoriality: let $f:X\to X'$ be a continuous map between Hausdorff spaces. Define $f^*:Z^{X'}\to Z^X$ by precomposition with $f$, i.e.
\[f^*(\psi:X'\to Z)=\psi\circ f:X\to Z.\]

This map is continuous: let $K\subset X$ be compact, $O\subset Z$ open, then
\[(f^*)^{-1}(W(K,O))=\cb{\psi:X'\to Z\mid (\psi\circ f)(K)\subset O}=W(f(K),O).\]
Since $K$ is compact, $f(K)$ is quasi-compact; since $X'$ is Hausdorff, $f(K)$ is compact.

Covariant functoriality: let $g:Z\to Z'$ be continuous. We define $g_*:Z^X\to (Z')^X$ by postcomposition with $g$, i.e.
\[g_*(\psi:X\to Z)=g\circ\psi:X\to Z'.\]

This map is continuous: let $K\subset X$ be compact and $O\subset Z'$ open, then
\[g_*^{-1}(W(K,O))=\cb{\psi:X\to Z\mid (g\circ\psi)(K)\subset O}=W(K,g^{-1}(O))\]
which is open in $Z^X$.

Note also that for $f:X\to X'$ continous, $g:Z\to Z'$ continuous and $X,X'$ Hausdorff, the following square commutes:
\begin{center}
    \begin{tikzcd}
    (Z')^{X'} \arrow[r,"f^*"] & (Z')^X\\
    Z^{X'} \arrow[u,"g_*"] \arrow[r,"f^*"] & Z^X \arrow[u,"g_*"]
    \end{tikzcd}
\end{center}
\end{remark}

\begin{example}
Let $f,g:X\to Z$ be homotopic maps, with $H:X\times[0,1]\to Z$ a homotopy between them. The adjoint $\Phi(H):[0,1]\to Z^X$ is then continuous. Hence homotopies correspond to paths in $Z^X$, so that the map:
\begin{align*}
[X,Z]&\to\pi_0(Z^X)\\
[f]\ \ &\mapsto\ \ [f]
\end{align*}
where $[X,Z]$ indicates the set of homotopy classes of maps $X\to Z$, is well-defined and a bijection whenever $X$ is locally compact.
\end{example}

\section{Path Spaces and Loop Spaces}

For a space $X$, the space $X^{[0,1]}$ is the \textbf{path space} of $X$. For a pointed space $(X,x_0)$, the \textbf{loop space} $\Omega X$ is the subspace of $X^{[0,1]}$ consisting of all loops at $x_0$, i.e. those $\omega\in X^{[0,1]}$ with $\omega(0)=\omega(1)=x_0$.

Let $q:[0,1]\to S^1=\cb{z\in\CC\mid|z|=1}$ be the quotient map $q(x)=e^{2\pi ix}$, this induces a continuous map
\[q^*:X^{S^1}\to X^{[0,1]},\]
which restricts to a continuous bijection onto $\Omega X$
\[(X,x_0)^{(S^1,1)}=\cb{f:S^1\to X\mid f(1)=x_0}.\]
This is in fact a homeomorphism, as a special case of the following lemma.

\begin{lemma**}
Let $q:X\to Y$ be a quotient map between compact spaces. Then for every space $Z$, the continuous map $q^*:Z^Y\to Z^X$ is a homeomorphism onto the subspace of all functions $f:X\to Y$ that factor through $q$.
\end{lemma**}
