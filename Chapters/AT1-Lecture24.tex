% Lecture 24

\lecture[]{2022-01-19}

An equivalence between $\Top[\weq^\inv]$ and $\Ho(\top_\text{CW})$.

The functor $|-|\circ\S:\Top\to\Top$ takes weak equivalences to homotopy equivalences and takes values in $\Top_\text{CW}$, so the composite
\[\Top\xto{|-|\circ\S}\Top_\text{CW}\xto{\text{proj}}\Ho(\Top_\text{CW})\]
inverts weak equivalences. So the universal property of the localizaton provides a unique functor
\[\lambda:\Top[\weq^\inv]\to\Ho(\Top_\text{CW})\]
such that the following square of categories and functors commutes
\[
\begin{tikzcd}
\Top \ar[d,"\gamma"] \ar[r,"\S"] & \sSet \ar[r,"{|-|}"] & \Top_\text{CW}\\
\Top[\weq^\inv] \ar[rr,dashed,"\exists!\Phi"] & & \Ho(\Top_\text{CW})
\end{tikzcd}
\]

\begin{theorem}
The functor $\Phi$ is an equivalence of categories.
\end{theorem}

\begin{proof}
On objects $\Phi$ is given by $\Phi(X)=|\S(X)|$, on morphisms $\Phi$ is
\[\Top[\weq^\inv](X,Y)=\Ho(\Top_\text{CW})(|\S(X)|,|\S(Y)|)\xto{\id}\Ho(\Top_\text{CW}(|\S(X)|,|\S(Y)|).\]
So $\Phi$ is fully faithful. We want to show that $\Phi$ is essentially surjective on objects. Let $K\in\Ho(\Top_\text{CW})$ be any object. Then the adjunction counit $\epsilon_K:|\S(K)|\to K$ is a weak equivalence, hence an homotopy equivalence by the Whitehead theorem, since source and target admit CW-structures. So the homotopy class of $\epsilon_K$ is an isomorphism in $\Ho(\Top_\text{CW})$, i.e. an isomorphism
\[\Phi(K)=|\S(K)|\xto{[\epsilon_K]}K?????\]
So $\Phi$ is an equivalence
\end{proof}

A morphism $f:X\to Y$ of simplicial sets is a \tbf{(simplicial) weak equivalence} if $|f|:|X|\to|Y|$ is a homotopy equivalence.

\begin{examples}
Homotopy equivalences of simplicial sets are weak equivalences. But weak equivalences are typically not homotopy equivalences.
Let $f:\de\Delta^2\to\Delta^1/\de\Delta^1$
be the morphisms such that $d_0,d_1\in(\de\Delta^1)_1$ map to the degenerate $1$-simplex of $\Delta^1/\de\Delta^1$ and maps $d_2$ to the non-degenerate $1$-simplex of $\Delta^1/\de\Delta^1$.\alvaropls

This $f$ is a weak equivalence but not a homotopy equivalence. Indeed, suppose $g:\Delta^1/\de\Delta^1\to\de\Delta^2$ is a morphism of simplicial sets, then $g$ must send the non-degenerate $1$-simplex $\id_{[1]}$ to a $1$-simplex of $\de\Delta^2$ where two vertices are the same: only the degenerate simplices have this property. So $|g|$ is a constant map at one of the $3$ vertices. In particular, $|g|$ is not a homotopy equivalence.

More generally, examples of weak equivalences that are not homotopy equivalences are the maps
\[\de\Delta^n\to\Delta^{n-1}/\de\Delta^{n-1}\]
that collapse one of the horns.
\end{examples}

\begin{proposition}
For every simplicial set $X$, the adjuntion unit $\eta_X:X\to\S|X|$ is a weak equivalence.
\end{proposition}

\begin{proof}
The triangle equality of the adjunction show that the composite
\[|X|\xto{|\eta_X|}|\S|X||\xto{\epsilon_{|X|}}|X|\]
is the identity. Then $|\eta_X|$ is a homotopy equivalence, hence an homotopy equivalence.
\end{proof}

Construction of a localization $\sSet[\weq^\inv]$.

The objects are all simplicial sets, the morphisms are $\Hom_{\Ho(\Top_\text{CW})(|X|,|Y|)}$. We define a localization:
\[\gamma:\sSet\to\sSet[\weq^\inv]\]
which is the identity on objects and $\gamma(f)=[(f)]$ on morphisms.

\begin{theorem}
The functor $\gamma:\sSet\to\sSet[\weq^\inv]$ is a localization at the class of weak equivalences.
\end{theorem}

\begin{proof}
We already know that $\gamma$ inverts weak equivalences. Now let $F:\sSet\to\Dd$ be any functor that inverts weak equivalences.

We will use a \enquote{calculus of fraction} for $\sSet[\weq^\inv]$: let $\alpha:|X|\to|Y|$ be any continuous map. Consider the commutative diagram of spaces
\[
\begin{tikzcd}
{|X|} \ar[d,"\alpha"] \ar[r,"\epsilon_{|X|}"] & {|\S|X||} \ar[d,"{|\S(\alpha)|}"]\\
{|Y|} \ar[r,"\epsilon_{|Y|}"] & {|\S|Y||}
\end{tikzcd}
\]

Problems!!!!!

The diagram
\[
\begin{tikzcd}
{|X|} \ar[d,"\alpha"] \ar[r,"{|\eta_{X}|}"] & {|\S|X||} \ar[d,"{|\S(\alpha)|}"]\\
{|Y|} \ar[r,"{|\eta_Y|}"] & {|\S|Y||}
\end{tikzcd}
\]
commutes up to homotopy. We pass to homotopy classes of maps and we get a commutative square
\[*****\]
hence $[\alpha]=\gamma(\eta_Y)^\inv\circ\gamma(\S(\alpha)\circ\eta_X)$.

Uniqueness. Let $G:\sSet[\weq^\inv]\to\Dd$ be any functor such that $G\circ\gamma=F$, then on objects
\[G(X)=G(\gamma(X))=F(X),\]
on morphisms
\[*****\]

Existence. Given $F:\sSet\to\Dd$, inverting equivalences, we define $G:\sSet[\weq^\inv]\to\Dd$ by $G(X)=F(X)$ on objects and $G[\alpha]=F(\eta_Y)^\inv\circ F(\S(\alpha))\circ F(\eta_X)$. This is well-defined, if $\alpha'\simeq\alpha:|X|\to|Y|$, then $F(\alpha')=F(\alpha)$... (this argument we did two times already)

Functoriality. Let $\alpha:|X|\to|Y|$ and $\ni:|Y|\to|Z|$ be continuous maps, $X,Y,Z$ simplicial sets. Then
\[*****\]
$G\circ\gamma=F$ similar to what we did for spaces.
\end{proof}

Equivalence between $\Top[\weq^\inv]$ and $\sSet[\weq^\inv]$.

The gr functor and the sc functor both preserve weak equivalences, hence they uniquely descend to the localizations as follows
\[
\begin{tikzcd}
\sSet \ar[d,"\gamma"] \ar[r,"{|-|}"] & \Top \ar[d,"\gamma"] \ar[r,"\S"] & \sSet \ar[d,"\gamma"]\\
\sSet[\weq^\inv] \ar[r,dashed,"\exists!\alpha"] & \Top[\weq^\inv] \ar[r,dashed,"\exists!\beta"] & \sSet[\weq^\inv]
\end{tikzcd}
\]

\begin{theorem}
The two composites $\beta\circ\alpha:\sSet[\weq^\inv]\to\sSet[\weq^\inv]$ and $\alpha\circ\beta:\Top[\weq^\inv]\to\Top[\weq^\inv]$ are naturally isomorphic to their respective identity functors. In particular, they are equivalence of categories
\end{theorem}

\begin{proof}
We give the argument only for $\beta\circ\alpha$. Composing the adjunction counit $\eta:\id_\sSet\to\S\circ|-|$ with the functor $\gamma:\sSet\to\sSet[\weq^\inv]$, yields a natural transformation
\[\gamma\circ\eta:\gamma\to\gamma\circ\S\circ|-|=\beta\circ\gamma\circ|-|=\beta\circ\alpha\circ\gamma.\]
The universal property of $\gamma$ for natural transformations provides a natural transformation $\tau:\id_{\sSet[\weq^\inv]}\to\beta\circ\alpha$ such that $\gamma\circ\eta=\tau\circ\gamma$. For every simplicial set $X$, this means that
\[\tau_X=\tau_{\gamma(X)}=\gamma(\eta_X)\]
is an isomorphism. So $\tau$ is a natural isomorphism.
\end{proof}

The previous argument applies very generally. *****

\begin{theorem}
The induced functor $\alpha:\Cc[\Ww^\inv]\to\Dd[\Vv^\inv]$ and $\beta:\Dd[\Vv^\inv]\to\Cc[\Ww^\inv]$ are equivalences.
\end{theorem}

\begin{corollary}
There is an equivalence of categories
\begin{align*}
    \sSet[\weq^\inv]&\cong\Ho(\Top_\text{CW})\\
    C&\mapsto|X|\\
    [\alpha:|X|\to|Y|]&\mapsto[\alpha]
\end{align*}
\end{corollary}

\begin{remark}
There is another category that is equivalent to the previous ones:
\[\Ho(\sSet_\text{Kan})\]
the homotopy category of \tbf{Kan-complexes}.
\end{remark}













