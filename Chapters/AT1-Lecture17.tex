% Lecture 17

\lecture[We finish to prove representability of singular cohomology. We show existence and uniqueness up to homotopy of CW-approximations to a topological space $Z$.]{2021-12-14}

\comment{
Last time we introduced the fundamental class of an EM-space $\iota\in H^n(K(A,n);A)$, the unique class such that the composite:
\[\pi_n(K(A,n),*)\xto{\cong} H_n(K(A,n);\Z)\xto{\Phi(\iota)}A\]
where the last morphism comes from the UCT, is the identification $\phi:\pi_n(K(A,n),*)\xto{\cong} A$.

For every CW-complex $X$, the map
\[[X,K(A,n)]\to H^n(X;A)\]
\[[f]\to f^*(\iota)\]
is a group homomorphism and the forgetful map $[X,K(A,n)]_*\to[X,K(A,n)]$ is bijective.

\begin{theorem}
For all $n\ge0$, all abelian groups $A$ and all CW-complexes $Y$, the evaluation homomorphism $[Y,K(A,n)]\to H^n(Y;A)$ is an isomorphism.
\end{theorem}
}

General case. We consider the cone of the $(\ni)$-skeleton, \[CY^{(\ni)}=(Y^{(\ni)}\times[0,1])/(Y^{(\ni)}\times\cb{1}),\]
and we form the CW-complex
\[Y\cup_{Y^{(\ni)}}CY^{(\ni)}.\]
Note that we have continuous maps
\[Y\xto{i} Y\cup_{Y^{(\ni)}}CY^{(\ni)}\xto{p}\Sigma Y^{(\ni)},\]
where $i$ is the inclusion and $p$ collapses $Y$.

We consider the commutative diagram of abelian groups:
\[
\begin{tikzcd}\label{diagram:representability-theorem}
{[\Sigma Y^{(\ni)},K(A,n)]} \ar[d] \ar[r,"p^*"] & {[Y\cup_{Y^{(\ni)}}CY^{(\ni)},K(A,n)]} \ar[d] \ar[r, "i^*"] & {[Y,K(A,n)]} \ar[d,"?"] \ar[r] & 0\\
H^n(\Sigma Y^{(\ni)};A) \ar[r,"p^*"] & H^n(Y\cup_{Y^{(\ni)}}CY^{(\ni)};A) \ar[r,"i^*"] & H^n(Y;A) \ar[r] & 0
\end{tikzcd}\tag{$*$}
\]
We want to use the $5$-lemma to show that the map on the right is an isomorphism.

Exactness of the upper row of (\ref{diagram:representability-theorem}). In the relative CW-complex $(Y\cup_{Y^{(\ni)}}CY^{(\ni)},Y)$ all relative cells have dimension less or equal to $n$. Since thee homotopy groups of $K(A,n)$ vanish up to dimension $\ni$, every continuous map $Y\to K(A,n)$ admits a continuous extension to $Y\cup_{Y^{(\ni)}}CY^{(\ni)}$. So the upper map $i^*$ is surjective. Since $p\circ i$ is the constant map, we have that $i^*\circ p^*=(p\circ i)^*$ is the zero homomorphism. Let $f:Y\cup_{Y^{(\ni)}}CY^{(\ni)}\to K(A,n)$ represent an element in the kernel of $i^*$, i.e. $f|_Y$ is nullhomotopic. The HEP for the pair $(Y\cup_{Y^{(\ni)}}CY^{(\ni)},Y)$ let us replace $f$ by a homotopic map $g:Y\cup_{Y^{(\ni)}}CY^{(\ni)}\to K(A,n)$ such that $g|_Y$ is the constant map at the basepoint:
\[
\begin{tikzcd}
Y \ar[dr,"\const_*"'] \ar[r,"i"] & Y\cup_{Y^{(\ni)}}CY^{(\ni)} \ar[d,"g"] \ar[r,"p"] & \Sigma Y^{(\ni)} \ar[dl,dashed,"h"]\\
& K(A,n) &
\end{tikzcd}
\]
So there is a unique continuous map $h:\Sigma Y^{(\ni)}\to K(A,n)$ with $h\circ p=g$. In particular, we have that $p^*[h]=[g]=[f]$, so that $\ker(i^*)=\im(p^*)$.

Exactness of the lower row of (\ref{diagram:representability-theorem}). This comes from the cohomology long exact sequence of the relative CW-complex $(Y\cup_{Y^{(\ni)}}CY^{(\ni)},Y)$. In particular, observe that the sequence would continue with
\[H^n(Y;A)\xto{\de} H^{n+1}(\Sigma Y^{(\ni)};A)\]
and $H^{n+1}(\Sigma Y^{(\ni)};A)=0$, since $\Sigma Y^{(\ni)}$ is a CW-complex of dimension less or equal to $n$.

The left vertical map in (\ref{diagram:representability-theorem}) is surjective. Since $\Sigma K(A,\ni)$ has trivial homotopy groups below the $n$-th and $K(A,n)$ above the $n$-th, using theorem \ref{theorem:correspondence-maps-and-group-morphisms} we can choose a continuous map, unique up to homotopy, $\kappa_n:\Sigma K(A,\ni)\to K(A,n)$, that realizes the following homomorphism on homotopy groups:
\[
    \pi_n(\Sigma K(A,\ni),*)\underset{\text{Hurewicz}}{\xto{\cong}}H_n(\Sigma K(A,\ni);\Z)\underset{\text{Suspension}}{\xto{\cong}} H_\ni(K(A,\ni);\Z)
\]
\[
    \underset{\text{Hurewicz}}{\xto{\cong}}\pi_\ni(K(A,\ni),*)\xto{\cong}A\xto{\cong}\pi_n(K(A,n),*).
\]

Then the following square commutes:
\[
\begin{tikzcd}[column sep=10em]
{[Y^{(\ni)},K(A,\ni)]} \ar[d,"\cong","{[f]}\mapsto f^*(\iota_\ni)"'] \ar[r,"{[f]}\mapsto{[\kappa_n\circ\Sigma f]}"] & {[\Sigma Y^{(\ni)},K(A,n)]} \ar[d,"{[f]}\mapsto f^*(\iota_n)"]\\
H^\ni(Y^{(\ni)};A) \ar[r,"\cong"] & H^n(\Sigma Y^{(\ni)};A)
\end{tikzcd}
\]
which gives us surjectivity of the map ${[f]}\mapsto f^*(\iota_n)$ (i.e. the left vertical map in (\ref{diagram:representability-theorem})), since ${[f]}\mapsto f^*(\iota_\ni)$ is an isomorphism by induction.

The middle vertical map in (\ref{diagram:representability-theorem}) is an isomorphism. This follows from the previous special case, since the space $Y\cup_{Y^{(\ni)}}CY^{(\ni)}$ is $(\ni)$-connected: indeed, by cellular approximation any continuous map $f:S^k\to Y\cup_{Y^{(\ni)}}CY^{(\ni)}$, with $k\le\ni$, is homotopic to a map with image the contractible space $CY^{(\ni)}$.

The $5$-lemma then shows that the right vertical map in (\ref{diagram:representability-theorem}) is an isomorphism.\qed

\begin{example}
$\rp{\infty}$ is both an EM-space of type ($\Z/2,1$) and a classifying space for real line bundles. So for any CW-complex $X$ we get two natural bijections
\[
    \Pic_\R(X)\xleftarrow{\cong}[X,\rp{\infty}]\xto{\cong}H^1(X,\FF_2),
\]
given by:
\[
    f^*(j)\mapsfrom [f]\mapsto f^*(\iota)
\]
where $j$ is the universal/tautological line bundle on $\rp{\infty}$. We can combine the two bijections into a map $w_1:\Pic_\R(X)\to H^1(X;\FF_2)$ which is called the first \tbf{Stiefel-Whitney class}.

One can show that a real line bundle is completely determined (up to isomorphism) by its first Stiefel-Whitney class (it is possible to see that vector bundles of higher rank are in general \tit{not} determined by their Stiefel-Whitney classes, though).

The complex version of this story uses $\cp{\infty}$:
\[
    \Pic_\CC(X)\xleftarrow{\cong}[X,\cp{\infty}]\xto{\cong} H^2(X,\Z),
\]
this gives a map $\Pic_\CC(X)\to H^2(X;\Z)$ which is called the first \tbf{Chern class}. Again, one can show that a complex line bundle over a CW-complex is determined up to isomorphism by its first Chern class (and that the same is not true in general for vector bundles of higher rank and their Chern classes).
\end{example}

\section{CW Approximation}

Our aim: we want to show that every space $Z$ admits a weak homotopy equivalence $X\to Z$ from a CW-complex $X$ that is unique up to homotopy.

A continuous map $f:X\to Y$ is a \tbf{weak homotopy equivalence}\rightnote{Being \enquote{weakly homotopy equivalent} is the equivalence relation generated by the relation \enquote{there is a weak homotopy equivalence between $X$ and $Y$}, which can be proved transitive but is not symmetric!} if $\pi_0(f):\pi_0(X)\to\pi_0(Y)$ is bijective and for all $n\ge1$ and all $x\in X$, $\pi_n(f):\pi_n(X,x)\to\pi_n(Y,f(x))$ is an isomorphism.\todo[color=yellow]{A bunch of interesting facts are in AT1Sheet10-2 and elsewhere, maybe add them}

Note that every homotopy equivalence is a weak homotopy equivalence (note: sometimes people (e.g. me) forget that this is not entirely trivial... see \cite[Proposition~1.18]{Hatcher} for a proof in the case of fundamental groups, which can be easily generalized to the higher homotopy groups).

For an example of a weak homotopy equivalence which is not an homotopy equivalence one could take the inclusion of a point into the long line or the Warsaw circle, or a map from a countable discrete space $Y$ to $\cb{0}\cup\cb{1/n\mid n\ge1}\subset\R$ with the subspace topology.\todo[color=yellow]{These pathological counterexamples can be somewhat subtle (even if perhaps ultimately not very interesting), I might add an appendix on them}

\begin{theorem}[Existence of CW-approximations]\label{theorem:cw-approximation}
Let $Z$ be a path connected space, $X$ a CW-complex, $x_0\in X$ a $0$-cell, $f:(X,x_0)\to(Z,z_0)$ a continuous map. Then there is a CW-complex $Y$ that contains $X$ as a subcomplex and a continuous extension $g:Y\to Z$ that is a weak homotopy equivalence.

As a special case, $X=\cb{x_0}$ gives the existence of CW-approximations.
\end{theorem}

\begin{proof}
This is similar to the proof of \ref{theorem:killing-homotopy-groups}, the method for "killing homotopy groups". We construct inductively CW-complexes $X=Y^{(-1)}\subset Y^{(0)}\subset Y^{(1)}\subset\cdots$ such that $Y^{(n)}$ contains $Y^{(\ni)}$ as a subcomplex and a continuous map $g^{(n)}:Y^{(n)}\to Z$ such that $g^{(n)}|_{Y^{(\ni)}}=g^{(\ni)}$ and $\pi_k(g^{(n)})=0$ for all $1\le k\le n$. Then $Y=\cup_{n\ge0}Y^{(n)}$ with the weak topology is the desired CW-complex and $g=\cup_{n\ge0}\,g^{(n)}:Y\to Z$ is the desired map because $\pi_k(g)=0$ for all $k\ge1$ by compactness\rightnote{A compact subspace of a CW complex is contained in a finite subcomplex. In case of doubt, see \cite[A.1]{Hatcher}\\ for a proof.}, hence $g$ is a weak homotopy equivalence by its associated long exact homotopy sequence.

Now, we proceed with the construction. Let $Y^{(-1)}=X$ and $g^{(-1)}=f:X\to Z$. Suppose that $Y^{(\ni)}$ and $g^{(\ni)}:Y^{(\ni)}\to Z$ with the desired properties have already been constructed. For each class $i\in\pi_n(g^{(\ni)})$ choose representing maps $\alpha_i:S^\ni\to Y^{(\ni)}$ and $\beta_i:D^n\to Z$ such that the diagram
\[
\begin{tikzcd}
S^\ni \ar[d,hook] \ar[r,"\alpha_i"] & Y^{(\ni)} \ar[d,"g^{(\ni)}"]\\
D^n \ar[r,"\beta_i"] & Z
\end{tikzcd}
\]
commutes. By cellular approximation we can assume that the $\alpha_i$ are cellular maps.

We construct $Y^{(n)}$ by attaching $n$-cells to $Y^{(\ni)}$
using the $\alpha_i$'s as attaching maps. Similarly, we define $g^{(n)}$ taking the union of $g^{(\ni)}$ with the $\beta_i$'s. Then $Y^{(n)}$ contains $Y^{(\ni)}$ as a subcomplex and $g^{(n)}$ extends $g^{(\ni)}$.

It remains to show that $\pi_k(g^{(n)})=0$ for $1\le k\le n$. To this end we compare the long exact homotopy sequences of $g^{(\ni)}$ and $g^{(n)}$:
\[
\small
\begin{tikzcd}
\pi_k(Y^{(\ni)},x_0) \ar[d,"\incl_*"] \ar[r,"g^{(\ni)}_*"] & \pi_k(Z,z_0) \ar[d,eq] \ar[r] & \pi_k(g^{(\ni)}) \ar[d,"\incl_*"] \ar[r,"\de"] & \pi_{k-1}(Y^{(\ni)},x_0) \ar[d,"\incl_*"] \ar[r,"g^{(\ni)}_*"] & \pi_{k-1}(Z,z_0) \ar[d,eq]\\
\pi_k(Y^{(n)},x_0) \ar[r,"g^{(n)}_*"] & \pi_k(Z,z_0) \ar[r] & \pi_k(g^{(n)}) \ar[r,"\de"] & \pi_{k-1}(Y^{(n)},x_0) \ar[r,"g^{(n)}_*"] & \pi_{k-1}(Z,z_0)
\end{tikzcd}
\]
The fourth map is surjective for $k\le n$ by cellular approximation, hence the middle map is surjective by the $5$-lemma. Since $\pi_k(g^{(\ni)})=0$ for $1\le k<n$, we have that also $\pi_k(g^{(n)})=0$ for $1\le k<n$.

For $k=n$ the map $\incl_*:\pi_n(g^{(\ni)})\to\pi_n(g^{(n)})$ sends all elements to zero by design: given an $i\in\pi_n(g^{(\ni)})$ represented by $(\beta_i,\alpha_i)$, we have the diagram
\[
\begin{tikzcd}[row sep=large]
S^\ni \ar[d,hook] \ar[r,"\alpha_i"] & Y^{(\ni)} \ar[d,"g^{(\ni)}"' near start] \ar[r,hook] & Y^{(n)} \ar[d,"g^{(n)}"]\\
D^n \ar[r,"\beta_i"'] \ar[urr,dashed,crossing over] & Z \ar[r,eq] & Z
\end{tikzcd}
\]
where the dashed arrow is a characteristic map for the $i$-th cell. The existence of the diagonal filler means that the outer square represents the $0$ element in $\pi_n(g^{(n)})$\leftnote{Because then you can contract $D^n$.}, hence $\incl_*$ is surjective and the zero homomorphism, so $\pi_n(g^{(n)})=0$.
\end{proof}

\begin{theorem}[Uniqueness of CW-approximations]\label{theorem:uniqueness-cw-approximations}
Let $f:X\to Z$ and $g:Y\to Z$ be two CW-approximations, i.e. weak homotopy equivalences to a path-connected space $Z$ such that $X$ and $Y$ are CW-complexes. Then there is a homotopy equivalence $\psi:X\to Y$ such that the diagram:
\[
\begin{tikzcd}
X \ar[rd,"f"'] \ar[rr,"\psi"] & & Y \ar[dl,"g"]\\
& Z &
\end{tikzcd}
\]
commutes up to homotopy.
\end{theorem}

\begin{proof}
We consider the map $f\amalg g:X\amalg Y\to Z$. Then there is a CW-complex $W$ that contains $X\amalg Y$ as a subcomplex and a weak homotopy equivalence $\Phi:W\to Z$ with $\Phi|_X=f$ and $\Phi|_Y=g$.

Because $g$ and $\Phi$ are weak homotopy equivalences, so is the inclusion $i_Y:Y\into W$
\[
\begin{tikzcd}
Y \ar[rd,"g"'] \ar[rr,"i_Y"] & & W \ar[dl,"\Phi"]\\
& Z &
\end{tikzcd}
\]
Since $Y$ and $W$ are path-connected, $i_Y$ is a homotopy equivalence by Whitehead's theorem. Let $j:W\to Y$ be a homotopy inverse.

Interchanging the roles of $X$ and $Y$ shows that also $i_X:X\to W$ is a homotopy equivalence.
So $\psi:j\circ i_X:X\to Y$ is a homotopy equivalence. Moreover: \[g\circ\psi=g\circ j\circ i_X=\Phi\circ i_Y\circ j\circ i_X\simeq\Phi\circ i_X=f.\]
\end{proof}
