% Lecture 11

\lecture[We continue studying mapping spaces.]{2021-11-22}

We prove the lemma from last lecture.\rightnote{\Attention\ I definitely still need to digest lecture 11 and 12, don't count on everything I write.}

\begin{proof}
First, note that $q^*$ is injective, continuous and has the desired image by the universal property of the quotient topology. We show that $q^*$ is also open as a map onto its image. We let $K\subset Y$ be compact and $O\subset Z$ be open. Then
\[q^*(W(K,O))=\cb{g\circ q:X\to Z\mid g\in Z^Y,\ g(K)\subset O}=W(q^{-1}(K),O)\cap\im(q^*).\]
Because $K$ is compact and $Y$ Hausdorff, $K$ is closed in $Y$; so $q^{-1}(K)$ is closed in $X$. Since $X$ is compact, $q^{-1}(K)$ is again compact. So $W(q^{-1}(K),O)\cap\im(q^*)$ is open.
\end{proof}

\begin{example}
The map $q:[0,1]\to S^1=\cb{z\in\CC:|z|=1}$, $q(x)=e^{2\pi ix}$ is a quotient map, so
\[q^*:X^{S^1}\to\cb{f\in X^{[0,1]}\mid f(0)=f(1)}\]
is a homeomorphism. For any base point $x_0\in X$, this homeomorphism restricts to a homomorphism
\[(X,x_0)^{(S^1,1)}\to\Omega X.\]

If $(Y,y_0)$ is a pointed space, the pointed version of $[Y,Z]=\pi_0(Z^Y)$ yields a well-defined surjective map $[Y,Z]_*\to\pi_0((Z,z_0)^{(Y,y_0))}$. If $Y$ is locally compact, this is a bijection.

For $Y=S^1$, this gives a bijection $\pi_1(Z,z_0)\cong\pi_0((Z,z_0)^{(S^1,1)})=\pi_0(\Omega Y)$.
\end{example}

\section{Loop Shifts the Homotopy Groups}

Let $(Y,y_0)$ be a based space. The reduced suspension is the space:
\[\Sigma Y=\frac{Y\times[0,1]}{Y\times 0\cup\cb{y_0}\times[0,1]\cup Y\times 1}.\]

The quotient map $q:Y\times[0,1]\to\Sigma Y$ induces a continuous injection:
\[q^*:(Z,z_0)^{(\Sigma Y,*)}\to Z^{Y\times[0,1]}\]
whose image consists of all continuous maps $f:Y\times[0,1]\to Z$ that factor through the quotient map, i.e. such that $f(Y\times 0\cup\cb{y_0}\times[0,1]\cup Y\times 1)=\cb{z_0}$. If $Y$ is compact, then so are $Y\times[0,1]$ and $\Sigma Y$, and $q^*$ is even an homeomorphism onto its image:
\[(Z,z_0)^{(\Sigma Y,y_0)}\cong(\Omega Z,\const_{z_0})^{(Y,y_0)}.\]

In particular:
\[\Hom_{\Top_*}((\Sigma Y,*),(Z,z_0))\cong\Hom_{\Top_*}((Y,y_0),(\Omega Z,\const_{z_0})).\]

So the functors $\Sigma$ and $\Omega$ are adjoint endofunctors in the category of based spaces.

On path components, we obtain a bijection
\[[\Sigma Y,Z]_*\cong[Y,\Omega Z]_*.\]

For $Y=S^n$ the last bijection specializes to a bijection
\[\pi_{n+1}(Z,z_0)=[S^{n+1},Z]_*\cong[\Sigma S^n,Z]_*\cong[S^n,\Omega Z]_*=\pi_n(\Omega Z,\const_{z_0})\]
which is even a group homomorphism (as we will see).

\section{Mapping Spaces and Serre Fibrations}

\begin{theorem}
Let $Z$ be any space, $(X,A)$ a relative CW-complex with $X$ and $A$ finite CW-complexes.\rightnote{\upshape A slightly more general statement was mentioned but I missed it. I guess $X$ and $A$ need not be finite but something weaker?} Then the restriction map $\incl^*:Z^X\to Z^A$, $f\mapsto f|_A$ is a Serre fibration.
\end{theorem}

\begin{proof}
We show that $\incl^*$ has the HLP with respect to all CW-complexes $Q$. So we consider a lifting diagram
\[\begin{tikzcd}
Q\times0 \arrow[d,hook] \arrow[r,"f"] & Z \arrow[d,"\incl^*"] \\
Q\times[0,1] \arrow[r,"\Phi"] & Z^A
\end{tikzcd}\]
Since $X$ and $A$ are locally compact we can use the exponential law to adjoint $f$ and $\Phi$ to continuous maps
\[\til f:X\times Q\times0\to Z\]
\[\til\Phi:A\times Q\times[0,1]\to Z\]
The condition $\incl^*\circ f=\Phi|_{Q\times0}$ becomes the relation that $\til f$ and $\til\Phi$ coincide on $A\times Q\times0$.

We consider the glued maps
\[\til f\cup\til\Phi:X\times Q\times 0\cup A\times Q\times[0,1]\to Z.\]

Since $(X\times Q,A\times Q)$ is a relative CW-complex, it has the HEP so there is a continuous extension $H:X\times Q\times[0,1]\to Z$ that extends $\til f$ and $\til\Phi$. The adjoint $H^\#:Q\times[0,1]\to Z^X$ of $H$ then solves the original lifting problem.
\end{proof}

For the relative CW-complex $([0,1],\cb{0,1})$ the theorem says that the restriction map
\[Z^{[0,1]}\to Z^{\cb{0,1}}\cong Z\times Z,\ w\mapsto (w(0),w(1))\]
is a Serre fibration for every space $Z$.

We let $z_0\in Z$ be any base point. Define $EZ$ as the subspace of $Z^{[0,1]}$ of all paths that start at $z_0$.

Equivalently, $EZ$ is the pullback:
\begin{center}
    \begin{tikzcd}
    w \ar[d,mapsto] &[-6ex] EZ \arrow[d] \arrow[r,hook] & Z^{[0,1]} \arrow[d] &[-6ex] w \ar[d,mapsto] \\
    w(1) & Z \arrow[r,"{(z_0,-)}"] & Z\times Z & (w(0),w(1))
    \end{tikzcd}
\end{center}

Since Serre fibrations are stable under base-change, we conclude that the map $EZ\to Z$, $w\mapsto w(1)$ is a Serre fibration.

\begin{theorem}\label{theorem:EZ-contractible}
The space $EZ$ is contractible onto the constant path at $z_0$.
\end{theorem}

\begin{proof}
Let $H:[0,1]\times[0,1]\to[0,1]$ be a homotopy, relative $\cb{0}$, that contracts the interval onto $0$, e.g. $H(s,t)=s(1-t)$. Let
\[H^*:Z^{[0,1]}\to Z^{[0,1]\times[0,1]}\]
be the continuous induced map and
\[\til H^*:Z^{[0,1]}\times[0,1]\to Z^{[0,1]}\]
its adjoint, i.e. $\til H^*(w,t)(s)=w(H(s,t))$.

We observe that
\[\til H^*(w,t)(0)=w(H(0,t))=w(0).\]
So whenever $w(0)=z_0$ (i.e. $w\in EZ$), then also $\til H^*(w,t)(0)=z_0$. In other words, for all $t\in[0,1]$, the map $\til H^*(-,t)$ takes $EZ$ to $EZ$. So we can restrict $\til H^*$ to a continuous map $\til H^*:EZ\times[0,1]\to EZ$; this is the desired contracting homotopy:
\[\til H^*(w,1)(s)=w(H(s,1))=w(0)\]
so the homotopy $\til H^*$ ends in the constant map at $z_0$.

For all $w\in EZ$, $\til H^*(w,0)(s)=w(t(s,0))=w(s)$, so $\til H^*(w,0)=w$.
\end{proof}

Now we can see that $\pi_n(\Omega,*)\cong\pi_{n+1}(Z,z_0)$ is a group morphism.

\begin{proof}[Second proof/construction of the isomorphism $\pi_n(\Omega Z,*)\cong\pi_{n+1}(Z,z_0)$]

We have seen that the space $EZ=\cb{w\in Z^{[0,1]}\mid w(0=z_0)}$ is contractible.

The map $e:EZ\to Z$, $e(w)=w(1)$ is a Serre fibration, so for every point in $Z$, we get a long exact sequence of homotopy groups. For $z_0\in Z$, the fibre of $e$ at $z_0$ is $\Omega Z$; hence the long exact sequence of homotopy groups is:
\[\dots\to\pi_{n+1}(EZ,*)=0\xto{e_*}\pi_{n+1}(Z,z_0)\xto{\de}\pi_n(\Omega Z,*)\xto{\incl_*}\pi_n(EZ,*)=0\to\cdots\]

So for $n\geq 1$, the connecting morphism $\de:\pi_{n+1}(Z,z_0)\to\pi_n(\Omega Z,*)$ is an isomorphism.
\end{proof}

\section{Turning Maps into Fibrations, up to Homotopy}

This is kind of dual to the mapping cylinder.

\begin{theorem}
Every continuous map $f:X\to Y$ can be factored functorially and naturally\todo[color=red]{ What does he mean by functorially and naturally?} as a composite
\[X\xto{\cong}Ef\xto{p} Y\]
of a homotopy equivalence followed by a Serre fibration.
\end{theorem}

\begin{proof}
Let $Ef=X\times_Y Y^{[0,1]}=\cb{(x,w)\in X\times Y^{[0,1]}\mid f(x)=w(0)}$. More precisely, $Ef$ is the pullback:
\[\begin{tikzcd}
Ef \arrow[d] \arrow[r] & Y^{[0,1]} \arrow[d] &[-6ex] w \ar[d,mapsto] \\
X \arrow[r,"f"] & Y & w(0)
\end{tikzcd}\]

We define natural continuous maps $h:X\to Ef$ by $h(x)=(x,\const_{f(x)})$ and $p:Ef\to Y$ by $p(x,w)=w(1)$.

Clearly we have $f=p\circ h$.

We observe that $Ef$ can be described as a slightly different pullback:
\[\begin{tikzcd}
(x,w) \ar[d,mapsto] &[-6ex] Ef \arrow[d] \arrow[r] & Y^{[0,1]} \arrow[d] &[-6ex] w \ar[d,mapsto]\\
(x,w(1)) & X\times Y \arrow[r,"f\times\id"] & Y\times Y & (w(0),w(1))
\end{tikzcd}\]

Since Serre fibrations are stable under base change, the map $Ef\to X\times Y$, $(x,w)\mapsto(x,w(1))$ is a Serre fibration. Since $\pr_2:X\times Y\to Y$ is a Serre fibration and Serre fibrations are closed under composition, $p$ is a Serre fibration. 
The map $h:X\to Ef$ is a homotopy equivalence: the homotopy inverse, which is also a left inverse, is the projection to the first factor.

Claim. The composite $c:Ef\to Ef$ is homotopic to the identity.

\begin{claimproof}
We define the desired homotopy
\[\bar H:Ef\times[0,1]\to Ef=X\times_Y Y^{[0,1]}\]
by specifying its projections to $X$ and to $Y^{[0,1]}$. The first coordinate of $\bar H$ is $Ef\times[0,1]\to X$, $(x,w,t)\mapsto x$, i.e. the constant homotopy of the projection to $X$. The second coordinate is the composite
\[Ef\times[0,1]\xto{\pr_2\times\id} Y^{[0,1]}\times [0,1]\xto{\til H^*} Y^{[0,1]}\]
where $\til H^*$ is the homotopy we constructed in the proof of theorem \ref{theorem:EZ-contractible}.

This has the following properties:
\begin{itemize}[label={-}]
    \item $\bar H$ starts with the identity,
    \item for all $t\in [0,1]$, all $w\in Y^{[0,1]}$, the path $\til H^*(w,t)$ has the same startpoint as $w$, so $\bar H$ really lands in $Ef$.
    \item $\til H^*(w,1)$ is constant at $w(0)$, which means that $\bar H(-,1)=h\circ\pr_1$.
\end{itemize}
\end{claimproof}
\end{proof}
