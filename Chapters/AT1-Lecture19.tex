% Lecture 19

\chapter{Spaces and Simplicial Sets: an Introduction to Homotopy Theory (?)}\label{chapter:the-cool-chapter}

\lecture[A lot of simplicial stuff, pt.2 (is this even topology anymore?).]{2021-12-20}

The long term goal for the rest of the semester is to show that there is an equivalence of homotopy theories (whatever this means):
\[(\Top,\text{weak eq.})\simeq(\sSet,\text{weak eq.}).\]

It is possible to prove (but it is beyond the scope of this course) that in fact there is an underlying equivalence of model categories/infinity categories (whatever this means): we will only show a "shadow" of this result, i.e. that there is an equivalence of homotopy categories (whatever this means).\todo{I don't know what I'm talking about, I'll rewrite this introduction in some weeks}

The short term goal for this lecture and the next one is to show the existence of a preferred\rightnote{In this course, we do not like the word \enquote{canonical}.} CW-structure on $|X|$, the geometric realization of a simplicial set $X$.

\section{The Preferred CW-structure on the Geometric Realization}
Recall the \hyperref[section:reminder-on-sset]{basic theory of simplicial sets} as exposed in chapter I.

The geometric realization of a simplicial set is the space:
\[|X|=\left(\coprod X_n\times\ns\right)/\sim\]
where $X_n$ is endowed with the discrete topology and the equivalence relation is generated by:
\[X_m\times\nabla^m\cont(x,\alpha_*(t))\sim(\alpha^*(x),t)\in X_n\times\nabla^n\quad\text{for all }\alpha:[n]\to[m],\ x\in X_m,\ t\in\nabla^m.\]

Note that we did not give explicitly the equivalence relation $\sim$ on $|X|$, just a relation (not symmetric, in particular) that \tit{generates} it,
hence it might be difficult to tell when two points of $\coprod_{n\ge0} X_n\times\ns$ represent the same equivalence class in $|X|$.
To solve this issue, we need to study the equivalence relation $\sim$ more in detail.

\begin{remark*}
Given any relation $R$ on a set $X$, the equivalence relation generated by $R$ is the intersection of all the equivalence relations containing $R$, i.e. $a\sim b$ if and only if there exists $\xs\in X$ such that $a=x_0$, $b=x_n$ and $x_iRx_{i-1}$ or $x_{i-1}Rx_i$ for all $i=0,\dots,n$.
\end{remark*}

\subsection{Minimal Representatives}\label{subsection:minimal-representatives}

An important feature of the relation $\sim$ is that classes have minimal representatives, as shown by the following proposition.

\begin{proposition}[Minimal representatives]\label{proposition:minimal-representatives}
Let $X$ be a simplicial set.
\begin{numerate}
\item Every equivalence class for $\sim$ has a unique representative $(x,t)\in X_l\times\sx{l}$ of minimal dimension $l$, called the \tbf{minimal representative}.
\item A pair $(y,s)\in X_n\times\ns$ is the minimal representative in its class if and only if:
\begin{itemize}[label={-}]
    \item $y$ is a non-degenerate simplex,
    \item $s$ is an interior point of $\ns$.
\end{itemize}
\item If $(x,t)\in X_l\times\sx{l}$ is the minimal representative in its class and $(y,s)\in X_n\times\ns$ is equivalent to $(x,t)$, then there is a unique triple $(\delta,\sigma,u)$ consisting of:
\begin{itemize}[label={-}]
    \item an injective morphism $\delta:[k]\into[n]$,
    \item a surjective morphism $\sigma:[k]\onto[l]$,
    \item $u\in\ring\nabla^{k}$,
\end{itemize}
such that $\delta^*(y)=\sigma^*(x)$, $s=\delta_*(u)$ and $t=\sigma_*(u)$.
\end{numerate}
\end{proposition}

Summing up: the first part of the proposition gives existence of minimal representatives, the second part a characterization of them and the third says that any element in an equivalence class is related to the minimal representative of the class by a chain of just two "elementary equivalences" (the relations by which the equivalence relation $\sim$ is generated), i.e. $(y,s)=(y,\delta_*(u))\sim(\delta^*(y),u)=(\sigma^*(x),u)\sim(x,\sigma_*(u))=(x,t)$.

\begin{proof}
We write $X_l^\text{nd}$ for the set of non-degenerate $l$-simplices. We first define a map
\[\rho:\coprod_{n\ge0} X_n\times\ns\to\coprod_{l\ge0} X_l^\text{nd}\times\ring\nabla^l\]
such that $\rho(y,s)$ is equivalent to $(y,s)$.

Consider any $(y,s)\in X_n\times\ns$. Suppose that $s=(\nno{s}{n})$. Since $\nnso{s}{+}{n}=1$ and all $s_i\ge0$, there is at least a coordinate which is positive. Suppose that $k+1$ of the coordinates are positive. Define $u=(\nno{u}{k})$ to be the coordinates of $s$ with the zero entries deleted, in the same order. We let $\delta:[k]\to[n]$ be the unique injective morphism such that $\delta_*(u)=s$. The simplex $\delta^*(y)\in X_k$ can be written uniquely as $\delta^*(y)=\sigma^*(x)$ for a surjective morphism $\sigma:[k]\to[l]$ and a non-degenerate simplex $x\in X_l^\text{nd}$, by proposition \ref{proposition:non-degenerate-simplices}. Then we set:
\[\rho(y,s)=(x,\sigma_*(u)).\]
Since $u\in\ring\nabla^{k}$ and $\sigma_*:\nabla^{k}\to\nabla^{l}$ adds coordinates together (hence does not insert zeroes), also $\sigma_*(u)\in\ring\nabla^l$. Hence the map $\rho$ is well-defined and by construction $\rho(y,s)\sim(y,s)$.

Claim. If $(y,s)\in X_n\times\ns$ and $(\bar y,\bar s)\in X_{\bar n}\times\sx{\bar n}$ are equivalent pairs, then $\rho(y,s)=\rho(\bar y,\bar s)$.

\begin{claimproof}
It suffices to show this when $(y,s)$ and $(\bar y,\bar s)$ are "elementary equivalent", i.e. there is a morphism $\alpha:[n]\to[\bar n]$ such that
\[(y,s)=(\alpha^*(\bar y),s)\sim(\bar y,\alpha_*(s))=(\bar y,\bar s).\]
We let $(\delta,u,\sigma,x)$ be the data in the construction of $\rho(y,s)$ and we choose a factorization (which is necessarily unique) $\alpha\circ\delta=\bar\delta\circ\bar\sigma$ for a surjective morphism $\bar\sigma:[k]\to[\bar k]$ and an injective morphism $\bar\delta:[\bar k]\to[\bar n]$. Then
\[\bar s=\alpha_*(s)=\alpha_*(\delta_*(u))=\bar\delta_*(\bar\sigma_*(u)).\]

Since $u\in\ring\nabla^{k}$ and $\sigma$ is surjective, $\bar\sigma_*(u)\in\ring\nabla^{\bar k}$. Hence $\bar s=\bar\delta_*(\bar\sigma_*(u))$ must be the unique expression in the first step of the construction of $\rho(\bar y,\bar s)$, since $\bar\delta_*$ is injective and $\bar\sigma_*(u)$ is an interior point. We write $\bar\delta^*(\bar y)=\hat\sigma^*(\hat x)$ for a surjective morphism $\hat\sigma:[\bar k]\to[\hat l]$ and a non-degenerate element $\hat x\in X_{\hat l}^\text{nd}$. Then we have:
\[\sigma^*(x)=\delta^*(y)=\delta^*(\alpha^*(\bar y))=\bar\sigma^*(\bar\delta^*(\bar y))=\bar\sigma^*(\hat\sigma^*(\hat x))=(\hat\sigma\circ\bar\sigma)^*(\hat x).\]
By the uniqueness of pairs of degeneracies and non-degenerate simplices (of proposition \ref{proposition:non-degenerate-simplices}), we must have $l=\hat l$, $x=\hat x$, $\sigma=\hat\sigma\circ\bar\sigma$. So $\bar\delta^*(\bar y)=\hat\sigma^*(\hat x)=\hat\sigma^*(x)$ and $(\bar\delta,\bar\sigma_*(u),\hat\sigma,x)$ must be the data in the construction of $\rho(\bar y,\bar s)$. Hence:
\[\rho(\bar y,\bar s)=(x,\hat\sigma_*(\bar\sigma_*(u)))=(x,\sigma_*(u))=\rho(y,s).\]
\end{claimproof}

(1) Suppose $(y,s)\in X_n\times\ns$ is of minimal dimension in its equivalence class. In the construction of $\rho(y,s)$ we must have $n\ge k\ge l$, therefore $n=k=l$, since $(y,s)\sim \rho(y,s)$. Clearly the injective map $\delta:[k]=[n]\into[n]$ and the surjective map $\sigma:[k]=[l]\onto[l]$ must be equal $\id_{[n]}$. Hence $\rho(y,s)=(y,s)$.

If $(\bar y,\bar s)$ is another representative of the same minimal dimension, then
\[(y,s)=\rho(y,s)=\rho(\bar y,\bar s)=(\bar y,\bar s).\]
This also shows that $\rho(y,s)$ produces the unique minimal representative.

(2) The necessity falls out of the construction of $\rho$, which produces the minimal representative of a given class. Conversely, suppose $y$ is non-degenerate and $s$ an interior point. If $s=\delta_*(u)$ for another interior point $u$, then $\delta$ must be the identity, but then $\delta^*(y)=y=\sigma^*(x)$, hence also $\sigma$ must be the identity, so $(y,s)=\rho(y,s)$ is the minimal representative.

(3) Suppose $(x,t)$ minimal, and $(y,s)$ equivalent (but not elementary equivalent) to it. We want to find a unique triple $(\delta,\sigma,u)$ of an injective map, a surjective map and an interior point such that $\delta^*(y)=\sigma^*(x)$, $s=\delta_*(u)$ and $t=\sigma_*(u)$. The existence comes from the construction of $\rho(y,s)$, while the uniqueness is because no choices were involved (every step is uniquely determined).
\end{proof}

While the previous proof is not exactly enlightening, having minimal representatives makes it easier to work with the geometric realization functor.

\begin{corollary}\label{corollary:criterion-for-injectivity-of-map-between-realizations}
Let $f:X\to Y$ be a morphism of simplicial sets such that $f_n:X_n\to Y_n$ is injective for all $n\ge0$.
\begin{numerate}
\item For every non-degenerate $x\in X_n$, the simplex $f_n(x)\in Y_n$ is non-degenerate.
\item The continuous map $|f|:|X|\to|Y|$ is injective.
\end{numerate}
\end{corollary}

Note: if $f$ has a retract, then (2) follows by functoriality, however, having a retract is much stronger than having injective components (the left inverses of the components might not assemble into a natural transformation).

\begin{proof}\ 

(1) Suppose that $f_n(x)=s_i^*(y)$ for some $0\le i\le\ni$, $y\in Y_\ni$. Then (using that $f$ is a morphism of simplicial sets) we have:
\[f_n(s_i^*(d_i^*(x)))=s_i^*(d_i^*(f_n(x)))=s_i^*(\underbrace{d_i^*(s_i^*(y))}_{\id_{[n]}^*(y)})=s_i^*(y)=f_n(x).\]
Since $f_n$ is injective, $s_i^*(d_i^*(x))=x$, which contradicts the assumption that $x$ is non-degenerate.

(2) Let $(x,t)\in X_n\times\ns$ and $(x',t')\in X_m\times\sx{m}$ be minimal representatives of classes in $|X|$ such that $|f|[x,t]=|f|[x',t']$. By definition of the geometric realization functor we have $|f|[x,t]=[f_n(x),t]$ and $|f|[x',t']=[f_m(x'),t']$, and by part (1) both $f_n(x)$ and $f_m(x')$ are non-degenerate.
Then by uniqueness of minimal representatives we must have $(f_n(x),t)=(f_m(x'),t')$, hence in particular $n=m$, $t=t'$ and $x=x'$, i.e. $[x,t]=[x',t']$.
\end{proof}

\begin{corollary}\label{corollary:compostition-of-realization}
Let $X$ be any simplicial set.
\begin{numerate}
\item The composite
\[\coprod_{n\ge0} X_n^\text{nd}\times\ns\into\coprod_{n\ge0} X_n\times\ns\onto|X|\]
is surjective.
\item The composite:
\[\coprod_{n\ge0} X_n^\text{nd}\times\ring\nabla^n\into\coprod_{n\ge0} X_n\times\ns\onto|X|\]
is a continuous bijection.
\end{numerate}
\end{corollary}

We will see that the second statement gives us the decomposition into cells of the preferred CW-structure for the geometric realization of a simplicial set.

We also have a corollary of (1).

\begin{corollary}\label{corollary:criterion-for-compactness-of-realization}
Suppose that the total number of non-degenerate simplices in all dimensions is finite. Then $|X|$ is quasi-compact.
\end{corollary}

\subsection{The Skeleta filtration}

The preferred CW-structure of $|X|$ arises from a "CW-like structure" that is intrinsic to the combinatorics of simplicial sets.

The \tbf{$m$-skeleton} $\sk^m X$ of a simplicial set $X$, for $m\ge0$, is the simplicial set defined by:
\[(\sk^m X)_n=\cb{x\in X_n\mid x=\alpha^*(y)\text{ for some }\alpha:[n]\to[m]\text{ and some }y\in X_m},\]
i.e. $\sk^m X$ is the smallest simplicial subset that contains $X_m$.

Clearly one has to convince himself that this is a simplicial set, which is not difficult.

\begin{example}
Every constant simplicial set is zero-dimensional, i.e. $X=\sk^0 X$. Conversely if $X=\sk^0 X$, then $X$ is isomorphic to the constant simplicial set on the zero-simplices $X_0$.
\end{example}

\begin{example}
For $\Delta^m=\Delta(-,[m])$, the simplicial $m$-simplex, we have $\sk^m(\Delta^m)=\Delta^m$ and $\sk^{m-1}(\Delta^m)=\de\Delta^m$, where $(\de\Delta^m)_n=\cb{\alpha:[n]\to[m]\mid\alpha\text{ not surjective}}$.
\end{example}

Note: these examples may not be necessarily obvious at first, one should diligently sit down and check the details.
