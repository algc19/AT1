% Lecture 20

\ 

\lecture[Finally we get to the preferred CW-structure on the geometric realization of a simplicial complex.]{2021-12-22}

The next proposition sheds some light on the definition of the skeleta of a simplicial set, showing that they are the right analogue of the skeleta of CW-complexes.

\begin{proposition}\label{proposition:getting-to-know-the-simplicial-skeleton}
let $X$ be a simplicial set.
\begin{numerate}
\item For $n\le m$, $(\sk^m X)_n=X_n$.
\item For $n>m$, all simplices of $(\sk^m X)_n$ are degenerate.
\item $\sk^m X\subset \sk^{m+1} X$.
\item $X$ is a colimit, in the category of simplicial sets, of the sequence:
\[\sk^0 X\subset \sk^1 X\subset\cdots\subset \sk^m X\subset\cdots\]
\item For every morphism $f:X\to Y$ of simplicial sets, $f(\sk^m X)\subset\sk^m Y$.
\end{numerate}
\end{proposition}

\begin{proof}\ 

(1) Let $n\le m$. We choose an injective morphism $\alpha:[n]\to[m]$ and a surjective morphism $\sigma:[m]\to[n]$ such that $\sigma\circ\alpha=\id_{[n]}$. Then for all $x\in X_n$ we have:
\[x=\id_{[n]}^*(x)=\alpha^*(\underbrace{\sigma^*(x)}_{\in X_m}),\]
hence $x\in(\sk^m X)_n$.

(2) We suppose that $n>m$, $x\in (\sk^m X)_n$. We write $x=\alpha^*(y)$ for some $\alpha:[n]\to[m]$, $y\in X_m$. Since $n>m$, $\alpha$ cannot be injective, so $\alpha(i)=\alpha(i+1)$ for some $0\le i<n$. So $\alpha=\beta\circ s_i$ for some morphism $\beta:[\ni]\to[n]$. Then
\[x=\alpha^*(y)=s_i^*(\beta^*(y))\]
is degenerate.

(3) We consider $x\in(\sk^m X)_n$, so $x=\alpha^*(y)$ for some $\alpha:[n]\to[m]$, $y\in X_m$. Then
\[x=\alpha^*(y)=(s_0\circ d_0\circ\alpha)^*(y)=(d_0\circ\alpha)^*(\underbrace{s_0^*\,(y)}_{\in X_{m+1}}),\]
hence $x\in(\sk^{m+1} X)_n$.

(4) Limits and colimits in functor categories are computed objectwise (a proof can be found in \cite[Theorem V.5.1]{maclane:71}, or in the nLab as the statement that \href{http://nlab-pages.s3.us-east-2.amazonaws.com/nlab/show/limits+of+presheaves+are+computed+objectwise}{limits of presheaves are computed objectwise}). Hence it suffices to show that in every fixed dimension $n$ the set is a colimit of the sequence:
\[(\sk^0 X)_n\into (\sk^1 X)_n\into\cdots\into (\sk^m X)_n\into\cdots\]
but by (1) this diagram stabilizes from $(\sk^n X)_n=X_n$ upwards, so $X_n$ is a colimit.

(5) Suppose $x=\alpha^*(y)\in(\sk^m X)_n$, for $\alpha:[n]\to[m]$, $y\in X_m$. Then
\[f_n(x)=f_n(\alpha^*(y))=\alpha^*(\underbrace{f_m(y)}_{\in Y_m}),\]
hence $x\in(\sk^m Y)_n$.
\end{proof}

\begin{remark}
Thanks to (5) of the previous proposition, we can make $\sk^m$ into an endofunctor of $\sSet$ by setting:
\[\sk^m f:=f|_{\sk^m X}:\sk^m X\to\sk^m Y.\]
Then the inclusions $\sk^m\into X$ and $\sk^m X\into\sk^{m+1} X$ define natural transformations $\sk^m\to\id$ and $\sk^m X\to\sk^{m+1}$ of endofunctors of $\sSet$.
\end{remark}

The following proposition (\ref{proposition:cw-structure-on-realization-preparation}) contains the technical work needed to show that the geometric realization of a simplicial set is a CW-complex with canonical CW-structure given by the realization of the simplicial skeleta (the statement is slightly more general and deals with generic simplicial subsets, we will specialize the result in theorem \ref{theorem:cw-structure-on-realization}).

\begin{remark}
In order to state and prove proposition \ref{proposition:cw-structure-on-realization-preparation} we need some preparation. First, we observe that there exists a natural transformation
$x^\flat:\Delta^m\to X$, that we call the \tbf{characteristic morphism} of $x\in X_m$, determined via the Yoneda lemma by $x^\flat_m(\id_{[m]})=x$ (hence $x^\flat_n(\alpha)=\alpha^*(x)$). Second, in the hypotheses of the proposition, we observe that the following square commutes (by naturality of the inclusion of the skeleton):
\[
\begin{tikzcd}
\de\Delta^m \ar[d,"x^\flat|_{\de\Delta^m}"] \ar[r,eq] & \sk^{m-1}\Delta^m \ar[d,"\sk^{m-1}x^\flat"] \ar[r,hook] & \Delta^m \ar[d,"x^\flat"]\\
\sk^{m-1} Y \ar[r,eq] & \sk^{m-1} X \ar[r,hook] & X
\end{tikzcd}
\]
In other words: the characteristic morphism of every $m$-simplex of $X$ sends $\de\Delta^m$ into the simplicial subset $Y$.
\end{remark}

\begin{proposition}\label{proposition:cw-structure-on-realization-preparation}
Let $X$ be a simplicial set, $m\ge0$. Let $Y$ be a simplicial subset of $X$ with $X_{m-1}=Y_{m-1}$. Suppose also that for $k>m$, every simplex in $X_k\sm Y_k$ is degenerate.
\begin{numerate}
\item The commutative square:
\[
\begin{tikzcd}[column sep=large]
\coprod_{x\in X_m\sm Y_m}\de\Delta^m \ar[d,"\amalg\,x^\flat|_{\de\Delta^m}"'] \ar[r,"\incl"] & \coprod_{x\in X_m\sm Y_m}\Delta^m \ar[d,"\amalg\,x^\flat"]\\
Y \ar[r,"\incl"] & X
\end{tikzcd}
\]
is a pushout of simplicial sets.
\item The commutative square of spaces:
\[
\begin{tikzcd}[column sep=large]
\coprod_{x\in X_m\sm Y_m}|\de\Delta^m| \ar[d,"\amalg\,|x^\flat||_{\de\Delta^m}"'] \ar[r,"|\incl|"] & \coprod_{x\in X_m\sm Y_m}|\Delta^m| \ar[d,"|\amalg\,x^\flat|"]\\
{|Y|} \ar[r,"|\incl|"] & {|X|}
\end{tikzcd}
\]
is a pushout of topological spaces.
\item The realization $|X|$ is obtained from $|Y|$ by attaching $m$-cells indexed by $X_m\sm Y_m$.
\end{numerate}
\end{proposition}

\begin{proof}\ 

(1) Since limits and colimits in the category of simplicial sets are computed objectwise (as we already noted in the proof of proposition \ref{proposition:getting-to-know-the-simplicial-skeleton}), it suffices that for every fixed dimension $n$, the commutative square of sets
\[
\begin{tikzcd}[column sep=large]
\coprod_{x\in X_m\sm Y_m}(\de\Delta^m)_n \ar[d,"\amalg\,x^\flat_n|_{\de\Delta^m}"'] \ar[r,"\incl"] & \coprod_{x\in X_m\sm Y_m}(\Delta^m)_n \ar[d,"\amalg\,x^\flat_n"]\\
Y_n \ar[r,"\incl"] & X_n
\end{tikzcd}
\]
is a pushout. Since the horizontal maps are injective, it suffices to show that the \enquote{complements} of the images of the horizontal maps are mapped bijectively by the right vertical map, i.e. that the right vertical map induces a bijection
\[\cb{X_m\sm Y_m}\times((\Delta^m)_n\sm(\de\Delta^m)_n)\to X_n\sm Y_n,\quad(x,\alpha)\mapsto\alpha^*(x).\]
By definition the $k$-simplices of $\de\Delta^m$ are all morphisms $\alpha:[n]\to[m]$ that are not surjective, hence we want to show bijectivity of the map
\[\cb{X_m\sm Y_m}\times\cb{\alpha:[n]\to[m]\mid\alpha\text{ surjective}}\to X_n\sm Y_n,\quad(x,\alpha)\mapsto\alpha^*(x).\]

If $n<m$, bijectivity is trivial because by hypothesis $X_n=Y_n$ (both sides are empty).

If $n=m$, bijectivity is immediate because we must have $\alpha=\id_{[n]}$.

If $n>m$, we can show bijectivity using proposition \ref{proposition:non-degenerate-simplices} and the hypothesis that all simplices in $X_k\sm Y_k$ are degenerate for $k>m$. In particular, injectivity follows from uniqueness of the representation of a simplex as a degeneracy of a non-degenerate simplex, since elements of $(X_m\sm Y_m)$ are non-degenerate (because otherwise they would be images of a simplex of dimension smaller than $m$, but in those dimensions $X$ and $Y$ are equal). To show surjectivity, take any simplex $x\in X_n\sm Y_n$ and write $x=\alpha^*(\bar x)$ for some surjective morphism $\alpha:[n]\to[k]$ and some non-degenerate simplex $\bar x\in X_k$. Since $x$ does not belong to $Y_n$, $\bar x$ does not belong to $Y_k$, hence we must have $k\ge m$. But we must also have $k\le m$, because otherwise all simplices in $X_k\sm Y_k$ would be degenerate. Hence $k=m$ and $(\bar x,\alpha)$ is an element of the domain with image $x$.

(2) The geometric realization functor $|-|:\sSet\to\Top$ is a left adjoint (to the singular complex functor $\S:\Top\to\sSet$), hence it preserves colimits\leftnote{A proof of this fact on the nLab: \href{http://nlab-pages.s3.us-east-2.amazonaws.com/nlab/show/adjoints+preserve+(co-)limits}{adjoints preserve (co-)limits}.} (in particular, finite ones, like coproducts and pushouts).

(3) From part (2) we know that $|X|$ can be obtained from $|Y|$ by attaching copies of $|\Delta^m|$ along $|\de\Delta^m|$, indexed by $X_m\sm Y_m$. So it suffices to show that the pair $(|\Delta^m|,|\de\Delta^m|)$ is homeomorphic to $(\sx{m},\de\sx{m})$, hence to $(D^m,S^{m-1})$.

Recall that we have mutually inverse homeomorphisms:
\begin{align*}
    |\Delta^m|\to\sx{m}&,\quad [\alpha:[n]\to[m],t\in\ns]\mapsto\alpha_*(t),\\
    \sx{m}\to|\Delta^m|&,\quad t\mapsto[\id_{[m]},t].
\end{align*}

We claim that these homeomorphisms map $|\de\Delta^m|$ homeomorphically onto the boundary of the topological simplex $\de\sx{m}=\cb{(\nno{t}{m})\in\sx{m}\mid t_i=0\text{ for some }0\le i\le m}$.

First, observe that the map $|\de\Delta^m|\to|\Delta^m|$ is injective by corollary \ref{corollary:criterion-for-injectivity-of-map-between-realizations} (which follows from the theory of minimal representatives). Secondly, since $\de\Delta^m$ has only finitely many non-degenerate simplices, $|\de\Delta^m|$ is compact by corollary \ref{corollary:criterion-for-compactness-of-realization} (which is also a consequence of the theory of minimal representatives). since $|\Delta^m|$ Hausdorff (being homeomorphic to $\sx{m}$), this is a closed map and hence a homeomorphism onto its image. Now, the non-degenerate simplices of $\de\Delta^m$ are all injective morphisms $\alpha:[k]\to[m]$ except $\id_{[m]}$ (which is surjective); by corollary \ref{corollary:composition-of-realization}, the image of $|\de\Delta^m|$ is then the union of the images of the maps $\alpha_*:\sx{k}\to\sx{m}$ for all injective morphisms $\alpha:[k]\to[m]$ other than $\id_{[m]}$, hence it coincides with $\de\sx{m}$, the boundary of $\sx{m}$.
\end{proof}

We are now ready to introduce the preferred CW-structure on the realization of a simplicial set, and summarize all its key properties, which follow from all the theory we have developed so far.

\begin{theorem}\label{theorem:cw-structure-on-realization}
Let $X$ be a simplicial set.
\begin{numerate}
\item The subspaces $|\sk^m X|$ for $m\ge0$ form a CW-structure on $|X|$, we will refer to it as the \tbf{preferred CW-structure} on the geometric realization of $X$.
\item The $m$-cells of the preferred CW-structure are in bijection with the set of the non-degenerate $m$-simplices of $X$.
\item Suppose that for all $n>m$, all $n$-simplices of $X$ are degenerate (i.e. $X=\sk^m X$). Then $|X|$ is an $m$-dimensional CW-complex.
\item Suppose that the total number of non-degenerate simplices of $X$ is finite. Then $|X|$ is a finite CW-complex. In particular, $|X|$ is compact.
\item For every morphism of simplicial sets $f:X\to Y$, the induced continuous map $|f|:|X|\to|Y|$ is cellular with respect to the preferred CW-structure.
\item If $Y$ is a simplicial subset of $X$, then $|Y|$ is a sub-complex of $|X|$ in the preferred CW-structure.
\end{numerate}
\end{theorem}

Note: another consequence of the theorem is that $|X|$ is Hausdorff. It might be not easy to show this directly.

\begin{proof}
We prove $(1)$ and $(2)$, the rest are immediate consequences.

Observe that we have $(\sk^{m-1} X)_{m-1}=X_{m-1}=(\sk^M X)_{m-1}$, and for every $n>m$, all simplices of $(\sk^n X)_m$ are degenerate. So $(\sk^m X,\sk^{m-1} X)$ satisfies the hypothesis of proposition \ref{proposition:cw-structure-on-realization-preparation}. Hence $|\sk^m X|$ can be obtained from $|\sk^{m-1} X|$ by attaching $m$-cells indexed by $(\sk^m X)_m\sm(\sk^{m-1} X)_m^\text{nd}$, which is the set of non-degenerate $m$-simplices of $X$.

Since $|-|$ is a left adjoint, it preserves colimits. Since $X$ is a colimit (in $\sSet$) of the sequence $\cb{\sk^m X}_{m\ge0}$, $|X|$ is a colimit (in $\Top$) of the sequence:
\[|\sk^0 X|\to|\sk^1 X|\to\cdots\to|\sk^m X|\to\cdots\]
Each of these maps is a cell attachment, hence in particular a closed embedding. So a colimit
of the sequence is the union with the weak topology.
\end{proof}

\begin{example}
We already know $|\Delta^m|\cong\sx{m}$ with the (obvious) CW-structure:
\begin{itemize}[label={-}]
    \item $(\sx{m})_0=\cb{\nno{e}{m}}$,
    \item $(\sx{m})_1=$ edges,
    \item $\cdots$,
    \item $(\sx{m})_{m-1}=\de\sx{m}$
    \item $(\sx{m})_m=\sx{m}$
\end{itemize}\ 

What about $|\Delta^m/\de\Delta^m|$? Intuitively this should be $S^m$ with the CW-structure consisting of one $0$-cell and one $m$-cell. To prove it, observe that $(\Delta^m)_n^\text{nd}=\cb{\alpha:[n]\to[m]\mid\alpha\text{ injective}}$ and $(\de\Delta^m)_n^\text{nd}=\cb{\alpha:[n]\to[m]\mid\alpha\text{ injective, but not surjective}}$. Then we have:
\begin{itemize}[label={-}]
    \item for $0\le n<m$, $(\Delta^m/\de\Delta^m)_n=\cb{*}$,
    \item for $n=m$, $(\Delta^m/\de\Delta^m)_m=\cb{\id_{[m]},*}$,
    \item for $n>m$, all simplices are degenerate.
\end{itemize}
This does not seem to give the CW-structure we predicted, but in fact it does: the only non-degenerate simplices are the one in $(\Delta^m/\de\Delta^m)_0$ and $\id_{[m]}\in(\Delta^m/\de\Delta^m)_m$.
\end{example}

\begin{example}
An interesting fact: we have $|\de\Delta^3|\cong S^2\cong|\Delta^2/\de\Delta^2|$ but as simplicial sets $\de\Delta^3\not\cong\Delta^2/\de\Delta^2$. Note that the CW-structure on $S^2$ resulting from realizing these two complexes is different, the one coming from $\de\Delta^3$ being much bigger (a tetrahedron: four $0$-cells, six $1$-cells, four $2$-cells).
\end{example}

\begin{example}
We have $|B\Z/2|\cong K(\Z/2,1)\cong\rp{\infty}$ (see the section in chapter \ref{section:construction-of-classifying-spaces}, although we skipped many details about classifying spaces), but the preferred CW-structure on $|B\Z/2|$ is different than the usual CW-structure on $\rp{\infty}$, with one cell in every dimension. Indeed, remembering the bar construction of AT1Sheet4, we see that the non-degenerate simplices in dimension $n$ are \[(B\Z/2)_n^\text{nd}=(\underbrace{1,\dots,1}_{n\text{-times}}),\] so the preferred CW-structure actually has one cell per dimension, but it is still not the same as the usual structure on $\rp{\infty}$ because the attaching maps behave differently (one can see this observing how attaching maps behave on the boundary of the cells, given the simplicial structure).\todo{Not really sure about why the structures are different}
\end{example}

\begin{example}\label{example:cw-bz-2}
We have $|B\Z|=K(\Z,1)\cong S^1$. Following the same reasoning as in the last example, we see that the non-degenerate simplices of $B\Z$ in any dimension are infinitely many. We already noted this fact in chapter \ref{example:cw-bz-1}: $S^1$ and $|B\Z|$ are homeomorphic CW-complexes, but the former is one-dimensional, the latter infinite-dimensional.
\end{example}
